%%%%% --------------------------------------------------------------------------------
%%
%%%%******************************* Main Content *************************************
%%
%%% ++++++++++++++++++++++++++++++++++++++++++++++++++++++++++++++++++++++++++++++++++


\section{学位论文选题的立论依据}

(课题来源、选题依据、理论与实际意义、预期研究成果的学术价值或应用价值等)

\subsection{课题来源}
自拟。


\subsection{选题依据}


复杂网络是建模复杂系统的重要手段\upcite{albert2002statistical,boccaletti2006complex,caldarelli2007scale},在复杂网络传统理论和技术方法当中,网络结构是重要的研究对象,被认为是导致系统整体功能涌现的主要因素。
随着信息技术的飞速发展,人们对于现实复杂系统的感知、认识、分析和处理进入了新的阶段,可获取的数据更精细、可运用的算力更强劲、挖掘信息的算法更智能,使得计算社会科学等以真实复杂系统为对象的研究能够不断深入和取得新进展。
然而,对于真实复杂网络结构信息的提取和运用仍然面临着诸多现实挑战。

首先是现实复杂网络的异质性,网络中各要素往往会表现出较大的差异。例如,在社会接触网络中,不同个体间的接触强度、接触模式甚至是接触接触行为不尽相同,且都会随着时间的推移而发生变化,因此难以用单一的静态图来表达其结构。

其次是现实复杂网络的层叠性,系统往往表现为多个涉及不同机理的网络共同作用的结果。例如,城市人口移动网络可能同时受到道路交通网络、轨道交通网络、社交网络、通讯网络等子网络的支配和影响,因此无法仅用某个层面的子结构来完整代表其整体功能特性。

再次是现实复杂网络的开放性,系统往往无法独立在一个封闭的环境中。例如,以文献引用网络通常无法同时涵盖所有的文献,社交媒体网络通常无法同时包括所有的社交用户,而只能以较为充分和独立的子网络为研究对象,因此无法彻底忽视网络外的连接对网络本身带来的影响。

最后是现实复杂网络的高阶性,系统中的交互模式往往不局限于一对一的交互。例如,信息在社交媒体上的传播通常是以群组和群发的方式进行,疾病的传播往往发生在人口密集的公共场所,而这些高阶的交互模式会使得网络结构的复杂性呈指数级增长。

基于上述真实复杂网络的现实特性,其结构往往是难以直接完整观测的,可观测的数据通常是其中各节点的属性状态随时间变化而形成的一组时间序列,并且序列之间可能存在相互影响和相互作用,共同构成一个完整的时空序列。
网络真实的网络结构就隐式的反映在序列间的相互作用中。
大多数的真实复杂网络动态,都可以用这样一种时空序列数据来描述,例如不同交通系统中不同地点记录的车流量、社交网络中不同用户的话题转发行为或活跃度、疾病传播网络中不同区域的受感染人群数量等。
因此,以系统历史时空序列数据为输入,来预测未来的系统演化行为的方法具有丰富且实际的应用场景。

现有的时空序列预测方法凭借深度学习模型强大的表示能力和端到端学习能力,能够在一些真实的任务场景中取得较好的预测效果,从而助力相关领域取得巨大的经济效益、激发产业活力。
然而,这些方法往往需要同时利用历史动态和网络结构作为输入,忽视了真实复杂网络难以直接完整观测的客观现实。
利用不完整甚至高度缺失的网络结构信息可能会成为制约模型预测性能的不利因素。
因此,如何在网络结构难以直接获得的现实约束下,对网络动态进行准确预测以及对隐式的网络结构进行推断是一个值得十分值得研究的课题。

\subsection{理论意义和应用价值}

本研究期望从动态复杂网络可观测的时空序列数据中提取出网络的结构特征,从而准确预测其未来的演化行为,并进一步通过推断来还原网络的结构。
就实际应用价值而言,本研究的成果相比于现有方法,将进一步提高预测的准确率,同时因为其不依赖网络结构作为输入,能够更好的适配各种不同场景不通约束的动态复杂网络预测任务。
同时,基于观测时空序列数据的网络重构方法能够显式还原出目标网络的结构,相比于预测的方法具有更好的可理解性和可解释性,进而在需要对复杂网络进行仿真模拟,以及调控和干预的任务场景中有着十分重大的应用价值。
就理论意义而言,本研究的成果或将揭示复杂网络结构影响复杂网络动态的客观规律,以及有关时空序列可预测性的科学问题。



%%% ++++++++++++++++++++++++++++++++++++++++++++++++++++++++++++++++++++++++++++++++++
