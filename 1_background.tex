%%%%% --------------------------------------------------------------------------------
%%
%%%%******************************* Main Content *************************************
%%
%%% ++++++++++++++++++++++++++++++++++++++++++++++++++++++++++++++++++++++++++++++++++


\section{学位论文选题的立论依据}

(课题来源、选题依据、理论与实际意义、预期研究成果的学术价值或应用价值等)

\subsection{课题来源}
自拟。


\subsection{选题依据}

随着信息技术的飞速发展,人们对于现实复杂系统的感知、认识、分析和处理进入了新的阶段,可获取的数据更精细、可运用的算力更强劲、挖掘信息的算法更智能,同时现实需求也对这些新技术提出了新的更高要求。

复杂网络是建模复杂系统的重要手段,在复杂网络传统理论和技术方法当中,网络结构是重要的研究对象,被认为是导致系统整体功能涌现的主要因素。
然而,现有的理论和技术对于现实复杂网络结构信息的提取和运用面临着诸多现实挑战。
首先是现实复杂网络的动态性,网络中各要素往往会随着时间变化,从而构成一个包含一定空间结构的时间序列,如何同时提取空间上的和时间上的信息来分析演化规律是现有数据挖掘算法面临的一个重要挑战;
其次是现实复杂网络的层叠性,系统往往表现为多个涉及不同机理的网络共同作用的结果,而人们对于网络结构的观测是十分有限的,如何在有限观测下还原系统整体功能和行为是一个重要挑战;
再次是现实复杂网络的异质性,现实中丰富的异质性细节给复杂网络结构的建模和分析方法的泛用性带来了很大的挑战;
最后是现实复杂网络的高阶性,高阶交互模式使得网络结构上的复杂度指数级增长,给传统网络结构分析方法的表示能力带来了极大的挑战。





\subsection{理论意义和应用价值}




%%% ++++++++++++++++++++++++++++++++++++++++++++++++++++++++++++++++++++++++++++++++++
