%%%%% --------------------------------------------------------------------------------
%%
%%%%******************************* Main Content *************************************
%%
%%% ++++++++++++++++++++++++++++++++++++++++++++++++++++++++++++++++++++++++++++++++++




\section{研究内容}

\subsection{研究目标}
复杂网络作为刻画现实系统中个体交互关系的核心工具,已广泛应用于生物调控、生态系统、社交传播、气候动力学等领域。
在人工智能时代,数据驱动的特征抽取和模式识别以及生成式智能体驱动的模拟仿真和演化预测已经成为各个领域研究科学问题的重要手段和未来趋势。
对现实复杂网络而言,其网络结构和演化动态是最直接也是最重要的两个研究对象,基于观测到的系统数据来掌握它们之间的作用机理和关联规律具有重要的理论意义和巨大的应用价值。

具体来说,我们需要预测或者模拟一个真实的复杂网络在未来的演化行为,从而能够最大限度的判断和评估该系统的一系列重要性质和一些关键节点并且能够借助计算机进行模拟推算。
更进一步,我们还需要知道驱动这个真实复杂网络演化的底层机理,从而能够实现有目的的干预。

然而,真实的复杂网络往往同时具有动态性、层叠性、异质性和高阶性,这使得我们难以直接从机理层面入手完整地建模该系统。
但随着人工智能技术的不断发展,我们可以基于观测到的历史数据,借助深度学习算法对演化过程进行拟合,从而实现掌握其局部演化规律的目的。
同时,真实复杂网络的演化机理往往也是复杂且多样的,甚至是人们难以理解的,就深度学习模型而言,这些机理蕴含在大规模的神经元参数中。
但网络的结构始终是决定系统行为的重要因素之一,而我们往往又无法直接观测到真实复杂网络的连接结构,因此基于观测数据来推断网络结构是掌握网络演化机理的重要手段。

\centerline{\rule[5pt]{\textwidth+2mm}{0.7pt}}% 表格中分割横线,勿删

\subsection{主要研究内容}


\subsubsection{基于弱监督的隐式网络结构表示学习方法研究}
图神经网络是目前对网络结构进行表征的主要手段之一,其通过图卷积操作将网络的结构信息映射到节点的特征空间表示当中。
统一的信息表征方式使得原本异构的网络结构信息能够很好的兼容各类下游信息抽取任务管道。
然而,图卷积网络只能针对显式的网络结构进行编码,在网络结构部分可观测甚至不可观测的情况下,图卷积网络将难以适用。
因此,如何有效利用弱监督信号并构造合适的代理任务管道对网络结构表征进行隐式的学习,是一个有意义的研究方向。

\subsubsection{网络结构缺失条件下的时空序列预测方法研究}
现有的时空序列预测方法通常需要同时利用时间维度上的时序变化信息以及空间维度上的网络结构信息。
然而,在许多现实场景下,我们难以准确观测到网络结构数据,甚至是无法观测。
因此,如何在网络结构缺失的情况下,仅利用各节点在时间维度上的时序变化数据,并充分挖掘各序列的相互依赖关系从而预测其接下来的变化趋势,是一个有意义的研究方向。

\subsubsection{基于时空序列数据的复杂网络重构方法研究}
在解决了前一个问题的基础上,将挖掘到的序列间相互依赖的关联信息应用于网络重构,将使得数据驱动的网络重构方法能够应用于大多数真实的任务场景。


\subsubsection{关于时空序列可预测性的理论研究}
对于单一的时间序列,可以基于熵来度量其可预测性。
但对于多个彼此关联的时间序列构成的时空序列,其可预测性是一个更为复杂的理论问题。
借助其他序列提供的外部信息,或可在单个序列的预测精度上超过单一序列的可预测性极限。
然而,外部信息并不总能为提升预测精度带来正向帮助,对单个时间序列的高精度预测也并不能保证对时空序列整体的准确预测。
因此,探究时空序列整体可预测性的度量指标以及网络结构、各节点时间序列在可预测性上关联关系,是一个有意义的研究方向。



\centerline{\rule[5pt]{\textwidth+2mm}{0.7pt}}% 表格中分割横线,勿删

\subsection{工作方案及可行性分析}

\subsubsection{研究方法}
本研究采取实证研究方法,以真实场景下的动态复杂网络为研究对象,例如交通运输数据、社交媒体转发数据、人口移动数据等。
基于现有的时空序列预测任务,通过实验验证隐式网络结构表示学习方法、网络结构缺失下的时空序列预测方法的有效性,并与当前SOTA方法进行比较。
基于现有的网络聚类、链路预测和网络重构方法,针对基于时空序列数据的网络重构任务设计实验流程和性能度量指标,验证重构方法的有效性并与基线模型进行比较。
最后,基于各实验结果进行统计分析,为时空序列可预测性的理论研究提供支撑。

\subsubsection{技术路线}
基于弱监督的隐式网络结构表示学习方法研究将围绕时空序列数据构造学习代理任务,并基于经典的神经网络模型架构搭建信息表征和特征抽取管道。



\centerline{\rule[5pt]{\textwidth+2mm}{0.7pt}}% 表格中分割横线,勿删
\subsection{预期创新点}




%%% ++++++++++++++++++++++++++++++++++++++++++++++++++++++++++++++++++++++++++++++++++
