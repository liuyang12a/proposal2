%%%%% --------------------------------------------------------------------------------
%%
%%%%******************************* Main Content *************************************
%%
%%% ++++++++++++++++++++++++++++++++++++++++++++++++++++++++++++++++++++++++++++++++++




\section{研究内容}

\subsection{研究目标}
社会复杂系统关系着人类生产生活的方方面面,是十分重要的研究对象。
对社会复杂系统的精准预测和高效模拟有着广泛且重要的现实意义,因此是一个十分热门的研究方向。
特别是随着人工智能时代的到来,高度自主的特征抽取和模式识别算法使得数据驱动的预测模型不断取得性能上的突破。
近年来,基于深度表示学习的预测技术在诸如人口移动、交通运输等场景中取得了一定的进展。
然而,广泛场景下的社会复杂系统预测和模拟方法还未出现,特别是针对人力资源管理领域等作用机理更复杂、数据缺失更严重、网络建模更困难的场景,现有方法的性能距离实际需求仍有较大差距。

具体来说,真实的社会复杂系统往往同时具有异质性、层叠性、开放性和高阶性,这使得网络化的建模过程中,难以直接获取其完整的网络结构。
而现有的预测方法高度依赖显式的网络结构作为输入从而获得嵌入表示,并用于后续的信息处理。
一方面,网络结构信息承载着系统中各节点的交互模式,蕴含着驱动系统演化的机理,因此对于预测十分重要;
另一方面,过度依赖输入的网络结构信息将导致模型在网络结构缺失的情况下难以适用,而且忽视结构的不完整性将给模型预测性能带来不利影响。

因此,针对真实社会复杂系统其网络结构难以直接完整获取的现实挑战,本研究的目标是基于深度表示学习框架的数据驱动端到端学习能力,改进现有方法存在的不足,
提出一种能广泛适应不同任务场景需求的社会复杂系统精准预测和高效模拟的解决方案。




\centerline{\rule[5pt]{\textwidth+2mm}{0.7pt}}% 表格中分割横线,勿删

\subsection{主要研究内容}


\subsubsection{基于弱监督的隐式网络结构表示学习方法研究}
图神经网络是目前对网络结构进行表征的主要手段之一,其通过图卷积操作将网络的结构信息映射到节点的特征空间表示当中。
统一的信息表征方式使得原本异构的网络结构信息能够很好的兼容各类下游的信息抽取任务管道。
然而,图卷积网络只能针对显式的网络结构进行编码,在网络结构部分可观测甚至不可观测的情况下,图卷积网络将难以适用。
因此,如何有效利用弱监督信号并构造合适的代理任务管道对网络结构表征进行隐式的学习,是本研究的一个重要的子内容。

\subsubsection{网络结构缺失条件下的时空序列预测方法研究}
网络结构承载着系统中各节点交互的模式信息,是影响系统演化动态的重要驱动因素之一。
网络结构的缺失将导致神经网络模型对系统演化规律的学习变得更加困难。
尤其是社会复杂系统,其网络系综往往会因任务场景的变化而发生变化,例如交通运输系统和人力资源系统,两者在规模、层次、密度上都有着显著的不同,结构驱动演化的机理也有所不同。
因此,在网络结构缺失时,如何设计模型架构和训练策略,实现从节点协同变化的时空序列中提取潜在的结构机理来提升预测性能,是本研究的一个重要的子内容。

\subsubsection{基于时空序列数据的复杂网络重构方法研究}
网络结构是认识、理解和分析一个复杂系统的重要切入点,尤其是在进行系统模拟和仿真乃至干预和控制时,网络结构是最为重要的输入。
然而对于大多数真实的社会复杂系统,其结构难以直接完整观测。
即使能够从各节点协同演化的时空序列中挖掘出相互作用的机理,但深度学习的黑盒模型仍然难以直接还原网络的结构。
因此,如何在现有的预测框架基础上,加上结构推断模块,设计高效的推断算法并利用白盒模型检验效果,是本研究的一个重要子内容。



\subsubsection{关于时空序列可预测性的理论研究}
对于单一的时间序列,可以基于熵来度量其可预测性。
但对于多个彼此关联的时间序列构成的时空序列,其可预测性是一个更为复杂的理论问题。
借助其他序列提供的外部信息,或可在单个序列的预测精度上超过单一序列的可预测性极限。
然而,外部信息并不总能为提升预测精度带来正向帮助,对单个时间序列的高精度预测也并不能保证对时空序列整体的准确预测。
因此,探究时空序列整体可预测性的度量指标以及网络结构、各节点时间序列在可预测性上关联关系,是一个有意义的研究方向。



\centerline{\rule[5pt]{\textwidth+2mm}{0.7pt}}% 表格中分割横线,勿删

\subsection{工作方案及可行性分析}

\subsubsection{研究方法}
本研究采取实证研究方法,以真实场景下的社会复杂系统为研究对象,例如交通运输数据、社交媒体转发数据、人口移动数据等。
基于现有的时空序列预测任务,通过实验验证隐式网络结构表示学习方法、网络结构缺失下的时空序列预测方法的有效性,并与当前SOTA方法进行比较。
基于现有的网络聚类、链路预测和网络重构方法,针对基于时空序列数据的网络重构任务设计实验流程和性能度量指标,验证重构方法的有效性并与基线模型进行比较。
最后,基于各实验结果进行统计分析,为时空序列可预测性的理论研究提供支撑。

\subsubsection{技术路线}
基于弱监督的隐式网络结构表示学习方法研究以及网络结构缺失下的时空序列预测方法研究,将围绕时空序列数据构造学习代理任务,并基于经典的神经网络模型架构搭建信息表征和特征抽取管道。
例如,Guo等人\upcite{guo2021learning}提出的用于时空序列预测的神经网络模型ASTGNN在交通流预测问题上展现出了优于其他同类方法的性能,其架构如图\ref{ASTGNN}所示:
\begin{figure*}[h]
    \centering
    \includegraphics[scale=0.5]{ASTGNN.PNG}
    \caption{ASTGNN 模型架构\upcite{guo2021learning}}     \label{ASTGNN}
\end{figure*}

ASTGNN参考了Transformer\upcite{vaswani2017attention}的结构设计,分为编码器和解码器两个部分。其中编码器用于从历史序列中抽取信息,解码器负责生成预测结果。
然而,与其他用于时空序列预测的模型类似,ASTGNN的输入包括代表着网络结构的邻接矩阵,并基于邻接矩阵计算对称归一化的拉普拉斯矩阵用于图卷积操作,从而获得节点的空间嵌入表征。
因此,在训练和推理阶段,表征网络结构的矩阵是固定不变的,图卷积核的参数会随着训练进行梯度更新,导致节点空间嵌入表征随之更新。
而隐式的网络结构表征学习方法,则认为网络结构是未知的,需要在训练过程中让模型直接学到合适的节点嵌入表征,不再需要网络结构矩阵作为输入。
相反,可以基于节点嵌入表征来生成对应的相似度矩阵作为输出,来近似网络的邻接矩阵,这在有网络结构作为标签时可以作为损失函数的一部分,作为正则化项来提供弱监督信号。

ASTGNN模型在使用图卷积层获取节点空间嵌入表示时,创新性的引入了一个稠密层用于自适应调整,其目的是认为交通网络结构在不同时刻对各节点流量影响也是动态的,并通过消融实验验证了该部分的有效性。
这恰恰也说明了显式的交通网络结构信息并不能代表全部的网络结构信息,即模型所利用的网络结构信息并不完整,是有缺失的。
基于ASTGNN模型改进的隐式网络结构表示学习方法,只是调整了模型的输入,并没有改变模型的输出,因此依然可以用于时空序列预测任务。
换言之,时空序列预测本身即作为指导模型学习网络结构表示的弱监督信号。
同时,整个模型管道也就是在解决网络结构缺失下的时空序列预测问题。

然而,ASTGNN在解决时空序列预测问题时,仍然还有值得改进的地方:
首先,就是依赖显式的网络结构信息作为输入,尽管使用了自适应调整的模块来模拟网络结构的动态变化,但机制过于简单,难以弥补结构缺失所导致的偏差。
其次,就是时域空域信息编码仍然缺少融合,其编码器解码器框架本身是为解决单个序列预测问题而提出的,迁移到多序列预测问题上时对于序列间交互机制依然缺少有效的表征和建模。
最后,就是在指标度量上只是度量了整个时空序列的预测精度,而没有仔细分析每个节点的预测精度,这可能会导致遗漏空域上的某些关键作用机制。
网络结构缺失的时空预测方法研究,将针对以上三个方面的问题进行改进,并提升预测性能。

基于时空序列数据的复杂网络重构方法研究将结合真实数据集和人工合成数据集,围绕链路预测、社群检测、图聚类等典型任务和方法,构建从时空序列数据推断网络结构的方法。
这里有两种不同的技术路线:一种是基于深度学习框架,搭建端到端的多任务学习框架,以时空数据为输入,让模型的输出能够拟合真实的网络结构。
难点在于有标注数据的获取、代理任务的配置以及目标函数的选择。
另一种是基于模拟仿真,通过手工设置的演化规则和网络结构进行模拟从而得到虚拟的时空序列数据,再基于虚拟的时空序列数据来反向推断网络结构,进而验证推断的效果。
难点在于构造合理且贴近真实情况的仿真试验平台。



\centerline{\rule[5pt]{\textwidth+2mm}{0.7pt}}% 表格中分割横线,勿删
\subsection{预期创新点}
本研究预期具有以下创新点:

一是针对现实场景中复杂网络的网络结构通常难以完全直接观测的问题,提出一种基于弱监督的隐式网络结构表示学习方法,能够在网络结构信息缺失的情况下从时空序列数据中自动提取空域位置嵌入表示,同时保持对各类下游任务的兼容性和泛化能力。

二是针对现有的时空序列预测方法普遍存在的依赖显式的网络结构信息、忽视信息残缺性以及对空域特征提取和利用不充分的问题,提出一种网络结构缺失下的时空序列预测方法,能够明显提升相关任务的预测精度。

三是针对真实复杂网络的动态、层叠、异质和高阶等特性以及难以完全观测所带来的挑战,提出一种基于时空序列数据的网络重构方法,能够在强噪声和数据缺失的情况下对特定的网络结构进行高效识别。

四是将时间序列可预测性的相关理论拓展延伸到时空序列预测任务场景,探索解决网络结构信息如何影响时空序列中不同节点演化可预测性的理论问题。



%%% ++++++++++++++++++++++++++++++++++++++++++++++++++++++++++++++++++++++++++++++++++
