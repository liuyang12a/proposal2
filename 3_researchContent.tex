%%%%% --------------------------------------------------------------------------------
%%
%%%%******************************* Main Content *************************************
%%
%%% ++++++++++++++++++++++++++++++++++++++++++++++++++++++++++++++++++++++++++++++++++




\section{研究内容}

\subsection{研究目标}
社会复杂系统关系着人类生产生活的方方面面,是十分重要的研究对象。
对社会复杂系统的精准预测和高效模拟有着广泛且重要的现实意义,因此是一个十分热门的研究方向。
特别是随着人工智能时代的到来,高度自主的特征抽取和模式识别算法使得数据驱动的预测模型不断取得性能上的突破。
近年来,基于深度表示学习的预测技术在诸如人口移动、交通运输等场景中取得了一定的进展。
然而,广泛场景下的社会复杂系统预测和模拟方法还未出现,特别是针对人力资源管理领域等作用机理更复杂、数据缺失更严重、网络建模更困难的场景,现有方法的性能距离实际需求仍有较大差距。

具体来说,真实的社会复杂系统往往同时具有异质性、层叠性、开放性和高阶性,这使得网络化的建模过程中,难以直接获取其完整的网络结构。
而现有的预测方法高度依赖显式的网络结构作为输入从而获得嵌入表示,并用于后续的信息处理。
一方面,网络结构信息承载着系统中各节点的交互模式,蕴含着驱动系统演化的机理,因此对于预测十分重要;
另一方面,过度依赖输入的网络结构信息将导致模型在网络结构缺失的情况下难以适用,而且忽视结构的不完整性将给模型预测性能带来不利影响。

因此,针对真实社会复杂系统其网络结构难以直接完整获取的现实挑战,本研究的目标是基于深度表示学习框架的数据驱动端到端学习能力,改进现有方法存在的不足,
提出一种能广泛适应不同任务场景需求的社会复杂系统精准预测和高效模拟的解决方案。




\centerline{\rule[5pt]{\textwidth+2mm}{0.7pt}}% 表格中分割横线,勿删

\subsection{主要研究内容}

\begin{figure*}[h]
    \centering
    \includegraphics[scale=0.5]{研究内容.PNG}
    \caption{现实挑战及研究内容}     \label{content}
\end{figure*}

\subsubsection{基于弱监督的隐式网络结构表示学习方法研究}
图神经网络是目前对网络结构进行表征的主要手段之一,其通过图卷积操作将网络的结构信息映射到节点的特征空间表示当中。
统一的信息表征方式使得原本异构的网络结构信息能够很好的兼容各类下游的信息抽取任务管道。
然而,图卷积网络只能针对显式的网络结构进行编码,在网络结构部分可观测甚至不可观测的情况下,图卷积网络将难以适用。
因此,如何有效利用弱监督信号并构造合适的代理任务管道对网络结构表征进行隐式的学习,是本研究的一个重要的子内容。

\subsubsection{网络结构缺失条件下的时空序列预测方法研究}
网络结构承载着系统中各节点交互的模式信息,是影响系统演化动态的重要驱动因素之一。
网络结构的缺失将导致神经网络模型对系统演化规律的学习变得更加困难。
尤其是社会复杂系统,其网络系综往往会因任务场景的变化而发生变化,例如交通运输系统和人力资源系统,两者在规模、层次、密度上都有着显著的不同,结构驱动演化的机理也有所不同。
因此,在网络结构缺失时,如何设计模型架构和训练策略,实现从节点协同变化的时空序列中提取潜在的结构机理来提升预测性能,是本研究的一个重要的子内容。

\subsubsection{基于时空序列数据的网络重构方法研究}
网络结构是认识、理解和分析一个复杂系统的重要切入点,尤其是在进行系统模拟和仿真乃至干预和控制时,网络结构是最为重要的输入。
然而对于大多数真实的社会复杂系统,其结构难以直接完整观测。
即使能够从各节点协同演化的时空序列中挖掘出相互作用的机理,但深度学习的黑盒模型仍然难以直接还原网络的结构。
因此,如何在现有的预测框架基础上,加上结构推断模块,设计高效的推断算法并利用白盒模型检验效果,是本研究的一个重要子内容。

\subsubsection{网络结构影响时空序列可预测性的机理研究}
对于单一的时间序列,其可预测性可以基于熵来进行度量。
但对于多个彼此关联的时间序列构成的时空序列,其可预测性是一个更为复杂的理论问题。
一方面,潜在的节点间交互寓示着时空序列存在一定的网络结构。
另一方面,网络结构的存在又意味着各时间序列不再具有独立性,将为单个时间序列预测引入可用的外部信息。
借助其他序列提供的外部信息,或可在单个序列的预测精度上超过单一序列的可预测性极限。
因此,探究时空序列整体可预测性的度量指标以及网络结构、各节点时间序列在可预测性上关联机理,是本研究的一个重要子内容。

\newpage
% \centerline{\rule[5pt]{\textwidth+2mm}{0.7pt}}% 表格中分割横线,勿删

\subsection{工作方案及可行性分析}

\subsubsection{研究方法}
本研究采取实证研究方法,以真实场景下的社会复杂系统为研究对象,例如交通运输数据、社交媒体转发数据、人口移动数据等。
基于现有的时空序列预测任务,通过实验验证隐式网络结构表示学习方法、网络结构缺失下的时空序列预测方法的有效性,并与当前SOTA方法进行比较。
基于现有的网络聚类、链路预测和网络重构方法,针对基于时空序列数据的网络重构任务设计实验流程和性能度量指标,验证重构方法的有效性并与基线模型进行比较。
最后,基于各实验结果进行统计分析,为时空序列可预测性的理论研究提供支撑。

\subsubsection{技术路线}

\begin{figure*}[h]
    \centering
    \includegraphics[scale=0.5]{研究框架.PNG}
    \caption{研究框架}     \label{arch}
\end{figure*}

首先,本研究将选择一些典型的社会复杂系统作为研究对象,收集相关任务场景中的数据用于实证研究和验证。
然后,本研究将构建一个总体研究框架,对问题的输入输出进行形式化的定义,并设计评估指标体系。
最后,本研究将在不同的任务数据集上进行实验分析,选取现有的一些SOTA方法作为基线,并根据实验结果进行针对性的改进和验证。

\paragraph{典型任务场景}
本研究将选择不同规模、不同尺度、不同数据条件的三个典型社会复杂系统作为研究对象,分别是人力资源系统、线上社交系统和交通运输系统。

在人力资源系统中,人才职业流动是一个基本且重要的社会现象。
近年来,网络化的职业路径分析以及人才流动动态过程正在吸引大量的研究关注。
职业流动的动态过程表现为各类人才在不同区域、不同组织、不同行业领域、不同层级上的迁移,其网络结构体现在承载人才的各单元之间的人才迁移规律之上。
对于这样一个人才职业流动网络,可观测的最基本的要素之一是人才简历。
一条就业轨迹,就是对职业流动动态的一次微观反映,综合大量的就业历史,就能够在宏观上追溯出一个庞大且复杂的人才流动网络。

在线上社交系统中,信息的传播是一个基本且重要的社会现象。
近年来,以谣言为代表的典型信息传播规律以及网络平台社群传播管控正在吸引大量的研究关注。
信息传播的动态过程表现为承载特定信息的某类媒体在不同用户之间的传递,其网络结构体现在信息传递渠道的“硬连接”以及群体认知偏好的“软连接”之上。
对于这样一个信息传播网络,可观测的最基本要素之一是用户的转发行为。
一次转发行为,就是对信息传播动态的一次微观反映,综合大量的转发历史,就能够在宏观上追溯出一个庞大且动态的信息传播网络。

在交通运输系统中,车流拥堵的传播是一个基本且重要的社会现象。
近年来,对车流量的预测和拥堵控制正在吸引大量的研究关注。
车流的动态反映为车辆在不同交通枢纽间的移动,其网络结构体现在道路交通网络的“硬连接”以及驾驶员行为意图的“软连接”之上。
对于这样一个客观的车流网络,观测手段较为丰富,其中一个重要的观测要素是散布在各交通枢纽节点处的传感器,能够实时检测该区域的车流情况。
传感器所处的空间位置,实际上就构成交通网络的节点,通过分析不同位置车流的协同变化,就能够大致还原出一个复杂且动态的车流网络。

\paragraph{总体研究框架}
本研究将假设社会复杂系统的网络结构信息缺失,仅以时空序列作为输入,从而尽可能的从中还原出体现节点间交互的网络连接信息,并对时空序列进行预测。
因此,本研究总体框架的基本输入定义如下:

确定的网络的节点集:$V=(v_1, v_2,...,v_n)$。

每个节点$v_i\in V$上确定的时间序列:$\xi_{:t}^{(i)}=\{..., \xi_{t-1}^{(i)}, \xi_{t}^{(i)}\}$。

根据实际任务的不同,本研究总体框架的输出也有所不同,对于基本的时空序列预测任务和网络重构任务,其基本输出定义如下:

每个节点在未来的时间序列预测:$\xi_{t+1:}^{(i)}=\{..., \xi_{t+1}^{(i)}, \xi_{t+2}^{(i)}\}$

节点间的交互结构:$E=\{(e,w)|e\in 2^{V} , w\in \mathbb{R}\}$,其中$e$表示潜在(高阶)连接,$w$表示连接强度(或置信度)。

考虑到一些社会复杂系统可观测的数据在时间维度和空间维度上并不能直接满足框架所需要的输入格式,因此需要进行统一的数据预处理。

当无法确定的网络节点集$V$时,需要先对$V$进行估计并构造。
对人力资源系统来说,$V$所指代的概念并不明确,可以是机构、也可以是岗位、或者其他的人力资源领域概念。
因此需要进行语义抽取和实体对齐,并完成对简历数据中每个条目的信息标注,从而得到全局统一的明确的网络节点集$V$。
对线上社交系统和交通运输系统来说,$V$所指代概念较为明确,因此不太需要额外的预处理流程来构造网络节点集。

当无法确定每个节点上的动态变化过程$\xi_{:t}^{(i)}$时,需要先对$\xi_{:t}^{(i)}$进行估计并构造。
对人力资源系统和线上社交系统来说,$\xi_{:t}^{(i)}$所指代的概念同样不能直接观测。
首先是信号强度,对人力资源系统来说,信号强度可以用节点的人才流入和流出来进行度量,并从简历数据中通过简单统计来得到;
对线上社交系统来说,信号强度可以用节点的转发频率来进行度量,并从转发行为数据中通过简单统计来得到。
而且,这样的构造方式,可以对统计时的时间采样窗口进行统一,从而对齐信号时序。
对于交通运输系统来说,传感器可以直接同步采集各处的时序信号,因此不需要额外的预处理。

\paragraph{实验分析验证}

本研究将选择当前主流的处理时空序列预测问题的SOTA方法作为基线,以一些典型的性能度量指标作为评价依据,通过实验验证方法效果并进行改进。

本研究将首先选取车流量预测问题作为切入点,因为该问题的概念定义明确且有大量公开的高质量数据集作为研究对象,因此目前有大量的时空序列预测方法是针对该问题进行的。
例如,Guo等人\upcite{guo2021learning}提出的用于时空序列预测的神经网络模型ASTGNN在交通流预测问题上展现出了优于其他同类方法的性能,其架构如图\ref{ASTGNN}所示:
\begin{figure*}[h]
    \centering
    \includegraphics[scale=0.5]{ASTGNN.PNG}
    \caption{ASTGNN 模型架构\upcite{guo2021learning}}     \label{ASTGNN}
\end{figure*}

ASTGNN参考了Transformer\upcite{vaswani2017attention}的结构设计,分为编码器和解码器两个部分。其中编码器用于从历史序列中抽取信息,解码器负责生成预测结果。
然而,与其他用于时空序列预测的模型类似,ASTGNN的输入包括代表着网络结构的邻接矩阵,并基于邻接矩阵计算对称归一化的拉普拉斯矩阵用于图卷积操作,从而获得节点的空间嵌入表征。
因此,在训练和推理阶段,表征网络结构的矩阵是固定不变的,图卷积核的参数会随着训练进行梯度更新,导致节点空间嵌入表征随之更新。
ASTGNN模型在使用图卷积层获取节点空间嵌入表示时,创新性的引入了一个稠密层用于自适应调整,其目的是认为交通网络结构在不同时刻对各节点流量影响也是动态的,并通过消融实验验证了该部分的有效性。
这恰恰也说明了显式的交通网络结构信息并不能代表全部的网络结构信息,即模型所利用的网络结构信息并不完整,可能会导致预测准确性降低。

又例如,Li等人\upcite{li2023dynamic}提出了一种名为 DGCRN 的交通预测框架,通过设计超网络实现节点属性的动态特征提取,并在每个时间步骤生成动态滤波器参数,从而过滤节点嵌入并生成动态图,与静态图集成,为该方法提供了高效性和性能。DGCRN 的架构如图\ref{DGCRN}所示:

\begin{figure*}[h]
    \centering
    \includegraphics[scale=0.4]{DGCRN.PNG}
    \caption{DGCRN 模型架构\upcite{li2023dynamic}}     \label{DGCRN}
\end{figure*}



以上方法虽然在一定程度上考虑到网络结构输入不完整的问题,但在解决网络缺失条件下的时空序列预测问题时,仍然还有值得改进的地方。

% 而隐式的网络结构表征学习方法,则认为网络结构是未知的,需要在训练过程中让模型直接学到合适的节点嵌入表征,不再需要网络结构矩阵作为输入。
% 相反,可以基于节点嵌入表征来生成对应的相似度矩阵作为输出,来近似网络的邻接矩阵,这在有网络结构作为标签时可以作为损失函数的一部分,作为正则化项来提供弱监督信号。
% 基于ASTGNN模型改进的隐式网络结构表示学习方法,只是调整了模型的输入,并没有改变模型的输出,因此依然可以用于时空序列预测任务。
% 换言之,时空序列预测本身即作为指导模型学习网络结构表示的弱监督信号。
% 同时,整个模型管道也就是在解决网络结构缺失下的时空序列预测问题。

(1)隐式网络表示学习

以上基线方法所提出的模型,在输入上都高度依赖显示的网络结构,在网络结构缺失的情况下难以适用。
本研究的第一个子内容就是要解决不依赖显式网络结构输入的隐式网络结构表示学习的问题。
一个基本的思路是用一个嵌入层来取代原本图卷积层的输出,从而绕过显式的网络结构来得到网络结构表示,再通过一系列的弱监督信号来训练嵌入层参数。
例如,对于ASTGNN解决的交通流预测问题,将原本直接作为输入的网络邻接矩阵用作标签配合一个结构生成模块来计算结构还原的损失函数,从而为网络结构表示提供弱监督信号。
将信息抽取的方式从直接抽取变为间接抽取。
同时,时空序列预测本身也可以作为训练隐式网络机构表示的弱监督信号。
而且,还可以组合借鉴对比学习、主动学习、数据增广等弱监督多任务训练技巧在不同领域中发挥积极作用的经验,探索出一套简答高效的隐式结构表示学习训练框架。

(2)提升时空序列预测性能

以上基线方法在预测的准确性上依然有改进空间。
本研究的第二个子内容就是要解决网络结构缺失条件下,时空序列预测性能受限的问题。
初步的思路仍然是改进和优化模型的网络结构表示学习部分,与第一个子内容不同的是,这里主要针对预测任务的性能而非兼顾多个弱监督任务。
既可以在使用第一个子内容获得的隐式结构表示,也可以加入一些其他的机制和组件来针对性地提高预测的准确性。

(3)网络重构

以上基线方法只是在生成对未来时空序列的预测,而没有考虑生成系统的网络结构。
本研究的第三个子内容就是要解决显式网络重构的问题。
一个初步的思路是人工模拟一个具有网络结构的系统演化动态,从而来检验网络重构方法的性能表现,对于网络重构方法的设计,主要还是以基于深度表示学习的生成式模型为主。

另一个思路是基于重构的网络,再设计模拟系统动态演化的白盒模型,从而检验白盒模型对系统动态的拟合性能。
例如,可以将信息的传播过程描述为各个个体受到某个消息影响的动态变化过程,从而用连续时间马尔科夫过程来作为一种统一且简洁的描述框架来模拟和预测传播的过程。

首先,用“在某个时刻是否受到影响”来作为每个节点在传播过程的状态描述,从而将节点的状态空间统一简化为概率描述,如图\ref{state}所示。
\begin{figure}
    \centering
    \includegraphics[scale=0.25]{状态表示.PNG}
    \caption{SIR模型的三状态简化为是否受影响的二状态。}     \label{state}
\end{figure}

然后,传播的动态可以描述为各节点受影响状态的连续变化,并使用密度矩阵$Q$来建模动力学特征,相当于是以$Q$矩阵所代表状态转移速率来建模节点间的传播强度,如图\ref{CTMC}所示。

\begin{figure}
    \centering
    \includegraphics[scale=0.4]{白盒模型.PNG}
    \caption{连续时间马尔科夫链作为描述传播过程的白盒模型。}     \label{CTMC}
\end{figure}

而网络重构的目标,就是对密度矩阵进行推理,因为$Q$矩阵本身就反映了传播过程中各节点相互影响的结构特征。
即通过重构网络从而得到一个能够预测传播过程的白盒模型。


(4)时空序列的可预测性

最后,以上基线模型只是在分析和比较时空序列整体的预测性能,而没有仔细分析系统中不同节点时间序列的预测性能。
本研究的第四个子内容就是要深入分析不同节点时间序列预测性能如何相互影响。
初步思路在本研究前三个子内容实验的基础上,深入分析神经网络模型在每个节点时间序列上的预测精度,以及网络结构的统计特征,再基于统计推断和假设检验的相关方法,来验证其中的一些关联关系。

例如,Zhang等人\upcite{zhang2022beyond}在人类移动轨迹预测方面的工作声称超越了可预测性的理论极限(如图\ref{beyond}所示),原因是其使用的神经网络模型经过了大量的人口移动数据的训练,从而使得模型在对单个人类移动轨迹进行预测时,具备了一定的上下文知识(例如相同社区的人在出行习惯上具有相似性,早上离开家会前往工作地点而晚上离开家则会前往购物区等,如图\ref{context}所示)。正是外部信息的引入使得模型的预测性能能够超越无额外信息时单个时间序列可预测性的极限。
\begin{figure}
    \centering
    \includegraphics[scale=0.5]{context.PNG}
    \caption{辅助预测的外部信息}     \label{context}
\end{figure}

\begin{figure}
    \centering
    \includegraphics[scale=0.5]{beyond.PNG}
    \caption{融合上下文信息的模型在预测性能上超越理论极限}     \label{beyond}
\end{figure}


类似的,在时空序列预测问题中,对多个彼此依赖的时间序列进行预测时,也可能会出现上述现象。
因为各时间序列将为彼此提供外部信息,从而导致有点时间序列可预测性强、而有的序列可预测性弱。
此外,与传统时间序列预测任务不同的是,时空序列预测因为要同时处理多个时间序列的信息,受限于计算效率和存储能力的限制,所能利用的历史信息在时间维度上通常远低于传统的时间序列预测任务。因此,空间维度上的来自彼此的外部信息对于预测性能就显得更加重要。




\newpage
% \centerline{\rule[5pt]{\textwidth+2mm}{0.7pt}}% 表格中分割横线,勿删
\subsection{预期创新点}
本研究预期具有以下创新点:

一是针对现实场景中复杂网络的网络结构通常难以完全直接观测的问题,提出一种基于弱监督的隐式网络结构表示学习方法,能够在网络结构信息缺失的情况下从时空序列数据中自动提取空域位置嵌入表示,同时保持对各类下游任务的兼容性和泛化能力。

二是针对现有的时空序列预测方法普遍存在的依赖显式的网络结构信息、忽视信息残缺性以及对空域特征提取和利用不充分的问题,提出一种网络结构缺失下的时空序列预测方法,能够明显提升相关任务的预测精度。

三是针对真实复杂网络的动态、层叠、异质和高阶等特性以及难以完全观测所带来的挑战,提出一种基于时空序列数据的网络重构方法,能够在强噪声和数据缺失的情况下对特定的网络结构进行高效识别。

四是将时间序列可预测性的相关理论拓展延伸到时空序列预测任务场景,探索解决网络结构信息如何影响时空序列中不同节点演化可预测性的理论问题。



%%% ++++++++++++++++++++++++++++++++++++++++++++++++++++++++++++++++++++++++++++++++++
