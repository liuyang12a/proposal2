%%%%% --------------------------------------------------------------------------------
%%
%%%%******************************* Main Content *************************************
%%
%%% ++++++++++++++++++++++++++++++++++++++++++++++++++++++++++++++++++++++++++++++++++




\section{学位论文工作计划}
{
\noindent
\begin{tabular*}{0.999\textwidth}{| p{0.20\textwidth } <{\centering} | p{0.30\textwidth}  | p{0.411\textwidth}  |}

	\hline
	\multicolumn{1}{|c|}{起讫日期} & 	\multicolumn{1}{c}{主要完成研究内容} & 	\multicolumn{1}{|c|}{预期成果} \\
	\hline
	\makecell{2025年07月- \\2025年09月}   &  深化文献调研 &   继续深入了解变分推断和心智模型的相关理论,打好数学基础 \\
	\hline
	\makecell{2025年10月-  \\2025年12月}   &  搭建实验环境 &   调研真实的人员画像任务场景,选取实验数据并筹备相应算力 \\
	\hline
    \makecell{2026年01月-  \\2026年03月}   &  研究点1 &  建模分析、实验验证  \\
	\hline
    \makecell{2026年04月-  \\2025年06月}   &  研究点1 &  形成论文  \\
	\hline
    \makecell{2026年07月-  \\2026年09月}   &  研究点2 &  建模分析、实验验证  \\
	\hline
    \makecell{2026年10月-  \\2026年12月}   &  研究点2 &  形成论文  \\
	\hline
    \makecell{2027年01月-  \\2027年03月}   &  研究点3 &  建模分析、实验验证  \\
	\hline
    \makecell{2027年04月-  \\2027年06月}   &  研究点3 &  形成论文  \\
	\hline
    \makecell{2027年07月-  \\2028年09月}   &  学位论文撰写 &  完成毕业论文的撰写和评审工作  \\
	\hline
    \makecell{2027年10月-  \\2027年12月}   &  毕业答辩 &  完成毕业答辩  \\
	\hline
	% \makecell{2018年07月 -- \\2018年08月} &  研究点1 &   发表论文SCI一篇 \\
	% \hline
	% \makecell{2018年09月 -- \\2018年10月} &  研究点2 &   完成实验 \\
	% \hline
    % \makecell{2018年11月 -- \\2018年12月} &  研究点2 &   发表论文EI一篇 \\
    % \hline
    % \makecell{2019年01月 -- \\2019年02月} &  研究点3 &   完成实验 \\
    % \hline
    % \makecell{2019年03月 -- \\2019年04月} &  研究点3 &   发表论文EI一篇 \\
    % \hline
    % \makecell{2019年05月 -- \\2019年06月} &  研究点4 &   完成实验 \\
    % \hline
    % \makecell{2019年07月 -- \\2019年08月} &  研究点4 &   发表论文EI一篇 \\
    % \hline
    % \makecell{2019年09月 -- \\2019年09月} &  研究点5 &   完成实验 \\
    % \hline
    % \makecell{2019年10月 -- \\2019年10月} &  研究点5 &   发表论文EI一篇 \\
    % \hline
	% \makecell{2019年11月 -- \\2020年01月} &  撰写毕业论文 &  完成毕业论文 \\
	% \hline
\end{tabular*}
\\[1 cm]
{\songti 注:每个子阶段不得超过3个月;预期成果中必须包含成果的形式、数量、质量等可考性指标该计划将作为
论文研究进展检查的依据。}
\indent
}


%%% ++++++++++++++++++++++++++++++++++++++++++++++++++++++++++++++++++++++++++++++++++
