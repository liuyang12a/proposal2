%%%%% --------------------------------------------------------------------------------
%%
%%%%******************************* Main Content *************************************
%%
%%% ++++++++++++++++++++++++++++++++++++++++++++++++++++++++++++++++++++++++++++++++++




\section{研究条件}

(开展研究应具备的条件及已具备的条件,可能遇到的困难与问题和解决措施。)

本研究的理论研究部分需要充分调研人员画像、贝叶斯推断和心智模型三个方面的理论、技术和方法,并对它们之间的联系和脉络加以梳理和总结。
目前,已经基本完成对人员画像技术和贝叶斯推断相关理论和方法的文献调研,并初步确定了在变分贝叶斯的框架下研究广义人员画像问题的技术路线。
在心智模型方面,主流的研究领域并没有形成一致公认的研究方向,且相关研究分散在心理学、生命科学、认知神经科学、计算神经科学、统计学等跨度较大的不同研究领域。
对心智模型的研究需要融合多个学科的背景知识以及较强的数学功底。
因此,需要在心智模型方面进一步拓展文献调研的广度和深度,充分了解各个领域的最新进展,同时进一步打牢数学基础。

本研究的实证研究部分需要选取真实人员画像需求牵引下的行为数据作为研究对象,以及适当的算力作为研究平台。
在实验数据选取方面,目前已经确定Kaggle上的ARC-AGI数据集作为研究内容3.模拟人类归纳能力的实验数据。对于其他研究内容,计划从课题任务现实需要出发,选取相关任务中的真实数据开展实验。
在平台搭建方面,目前自购了一张NVIDIA GeForce RTX3070Ti显卡,因为前三个研究内容实验中用到的主要是概率编程框架,且不涉及复杂的概率图和大规模张量计算,因此基本能够满足算力需要。

%%% ++++++++++++++++++++++++++++++++++++++++++++++++++++++++++++++++++++++++++++++++++
