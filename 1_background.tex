%%%%% --------------------------------------------------------------------------------
%%
%%%%******************************* Main Content *************************************
%%
%%% ++++++++++++++++++++++++++++++++++++++++++++++++++++++++++++++++++++++++++++++++++


\section{学位论文选题的立论依据}

(课题来源、选题依据、理论与实际意义、预期研究成果的学术价值或应用价值等)

\subsection{课题来源}



\subsection{选题依据}

社会复杂系统是人类社会活动的主要组织形式,大到国家和军队的人员组织和运转、小到某个消息在特定社群中传播,都表现出个体行为的随机性和群体行为的统计特性。
在对社会复杂系统的研究中,复杂网络是建模复杂系统的重要手段\upcite{albert2002statistical,boccaletti2006complex,caldarelli2007scale},
即用网络连接来表示系统各节点间的交互模式,从而将一个复杂的社会系统的演化过程建模为复杂网络上的动态过程。
例如,以用户群体为节点的信息传播网络、以地理空间位置为节点的人口移动网络、以交通枢纽为节点的运输网络以及以机构岗位为节点的人力资源网络。
对网络化的社会复杂系统演化过程进行预测、模拟乃至干预和控制具有重要的现实意义,研究者们围绕其开展了大量的研究工作,相关成果带来了显著的经济效益。

% 在复杂网络传统理论和技术方法当中,网络结构是重要的研究对象,被认为是导致系统整体功能涌现的主要因素。
% 随着信息技术的飞速发展,人们对于现实复杂系统的感知、认识、分析和处理进入了新的阶段,可获取的数据更精细、可运用的算力更强劲、挖掘信息的算法更智能,使得计算社会科学等以真实复杂系统为对象的研究能够不断深入和取得新进展。
然而,在处理真实的社会复杂系统时,依然存在着诸多的挑战,其中一个关键挑战在于,真实的网络结构难以直接完整获取:

首先是真实社会系统的异质性,系统各要素往往会表现出较大的差异。例如,在社会接触网络中,不同个体间的接触强度、接触模式甚至是接触接触行为不尽相同,且都会随着时间的推移而发生变化,因此难以用单一的静态图来表达其结构。

其次是真实社会系统的层叠性,系统往往表现为多个涉及不同机理的网络共同作用的结果。例如,城市人口移动网络可能同时受到道路交通网络、轨道交通网络、社交网络、通讯网络等子网络的支配和影响,因此无法仅用某个层面的子结构来完整代表其整体功能特性。

再次是真实社会系统的开放性,系统往往无法独立在一个封闭的环境中。例如,以文献引用网络通常无法同时涵盖所有的文献,社交媒体网络通常无法同时包括所有的社交用户,而只能以较为充分和独立的子网络为研究对象,因此无法彻底忽视网络外的连接对网络本身带来的影响。

最后是真实社会系统的高阶性,系统中的交互模式往往不局限于一对一的交互。例如,信息在社交媒体上的传播通常是以群组和群发的方式进行,疾病的传播往往发生在人口密集的公共场所,而这些高阶的交互模式会使得网络结构的复杂性呈指数级增长。

正因为真实社会系统同时存在的以上现实特性,几乎不可能获取到网络的全部结构细节,也就很难完全从机理层面入手建立其动态演化的模型并取得较好的预测性能。
因此,数据驱动的预测和仿真是当前的主流研究方向。
以深度学习为代表的数据挖掘算法在大数据的加持下,能够很好的克服特征工程难以构建、机理难以刻画的问题,在一些真实任务场景中取得较好的预测效果。
但相关研究仍然处在发展初期,以下方面仍需进一步探索和改进:

一是网络结构特征的提取高度依赖显式的网络结构输入。
在系统中各节点协同演化的过程中,网络结构是刻画其相互作用的重要信息,因此当前主流方法(如各类图神经网络)重点关注从已知的网络结构中提取特征从而用于预测。
但是,在以人类活动为基本要素的社会系统中,仅用单一静态的网络结构无法表征所有的交互细节。
因此传统的网络结构特征提取方法可能难以适应交互机制复杂多变的真实任务场景。

二是多任务场景下预测方法的性能和泛化能力还有待提升。
虽然目前基于深度学习的预测方法在预测精度上不断取得提升,但在长时预测、少样本预测、跨模式预测等更实际的预测需求上依然有非常大的提升空间。
而且,在不同的任务场景中,可获取的网络结构信息完整度不同,节点间交互模式不同,驱动系统演化的机理不同,导致各自的方法和策略难以通用。

三是从系统动态中推断出合理网络结构的方法还有待探索。
网络重构本身是一个十分具有挑战的任务,传统的网络重构方法大都聚焦于静态网络,即基于观测到的局部网络信息来推断网络整体。
而在实际的社会系统中,可获取的网络结构信息十分有限甚至不可观测,而要推断的网络系综又非常复杂,因此从系统动态来推断网络结构极具挑战。

四是网络结构如何影响系统可预测性的理论研究还有待深入。
在社会复杂系统的演化动态中,不仅蕴含着各节点演化的不确定性,同时还蕴含着节点相互作用的关联性。
现有理论基于演化的不确定性能够度量单个节点的可预测性,却没有考虑节点外的信息注入对可预测性带来的影响。
以及整体来看,节点间可预测性的关联关系,其中,网络结构扮演着重要的角色。
节点协同演化过程中,网络结构对系统整体可预测性带来的影响是一个十分值得研究的理论问题。


% 基于上述真实复杂网络的现实特性,其结构往往是难以直接完整观测的,可观测的数据通常是其中各节点的属性状态随时间变化而形成的一组时间序列,并且序列之间可能存在相互影响和相互作用,共同构成一个完整的时空序列。
% 网络真实的网络结构就隐式的反映在序列间的相互作用中。
% 大多数的真实复杂网络动态,都可以用这样一种时空序列数据来描述,例如不同交通系统中不同地点记录的车流量、社交网络中不同用户的话题转发行为或活跃度、疾病传播网络中不同区域的受感染人群数量等。
% 因此,以系统历史时空序列数据为输入,来预测未来的系统演化行为的方法具有丰富且实际的应用场景。

% 现有的时空序列预测方法凭借深度学习模型强大的表示能力和端到端学习能力,能够在一些真实的任务场景中取得较好的预测效果,从而助力相关领域取得巨大的经济效益、激发产业活力。
% 然而,这些方法往往需要同时利用历史动态和网络结构作为输入,忽视了真实复杂网络难以直接完整观测的客观现实。
% 利用不完整甚至高度缺失的网络结构信息可能会成为制约模型预测性能的不利因素。
% 因此,如何在网络结构难以直接获得的现实约束下,对网络动态进行准确预测以及对隐式的网络结构进行推断是一个值得十分值得研究的课题。

\subsection{理论意义和应用价值}

本研究着眼于真实社会系统特别是人力资源复杂系统演化、干预和控制的现实任务背景,综合利用复杂网络、深度学习、数据挖掘、因果推断等前沿理论及技术,
解决数据驱动的复杂网络演化动态预测和结构机理挖掘上的关键问题。
本研究的成果不仅能够在具体的现实任务场景中充分挖掘数据中潜在的机制和规律,有效提升预测的精准度,为决策者和管理层提供信息支援。
还能够在理论层面为复杂网络演化可预测性的科学问题提供实证路径和数据解释。
下面分别从理论意义和应用价值两个方面来阐述本研究的重要性:

(1)理论意义
从理论角度出发,本研究探索的基本科学问题是复杂系统演化过程中,节点交互所形成的网络结构对其演化动态的影响规律;以及从信息论的视角看,基于可观测的演化动态历史数据能够在多大程度上推断和还原网络结构信息。
具体到社会计算和人力资源管理领域背景中,本研究将能够丰富相关领域的理论研究成果。
以人力资源复杂系统和人才职业流动为例,传统的理论研究高度依赖专家知识,机理挖掘片面且主观性强;而数据驱动的方法在运用数据的过程中,因为网络结构难以被直接观测,其作为系统演化驱动要素的重要性和复杂性常被忽视。因此鲜有从网络结构的角度入手,分析人力资源复杂系统差异以及人才职业流动规律的理论,而本研究将有助于相关理论的建立和完善。
又比如在政策模拟和社会系统仿真领域,往往难以捕捉和掌握系统自发的演化动态,给相关理论的构建带来了较大的挑战,而本研究将从网络结构的角度对自发的演化动态提供一种精准且高效的刻画,从而降低相关理论构建的难度。

(2)应用价值
从应用角度出发,本研究要解决的重大现实问题是实现对真实社会复杂系统演化动态的精准预测和持续高效模拟。
本研究将基于深度表示学习框架提出一种泛用的数据驱动的社会复杂系统演化预测方法,解决现有预测方法高度依赖显式网络结构输入的问题,从而能够适应大多数网络结构缺失的任务场景。
同时,


% 期望从动态复杂网络可观测的时空序列数据中提取出网络的结构特征,从而准确预测其未来的演化行为,并进一步通过推断来还原网络的结构。
% 就实际应用价值而言,本研究的成果相比于现有方法,将进一步提高预测的准确率,同时因为其不依赖网络结构作为输入,能够更好的适配各种不同场景不通约束的动态复杂网络预测任务。
% 同时,基于观测时空序列数据的网络重构方法能够显式还原出目标网络的结构,相比于预测的方法具有更好的可理解性和可解释性,进而在需要对复杂网络进行仿真模拟,以及调控和干预的任务场景中有着十分重大的应用价值。
% 就理论意义而言,本研究的成果或将揭示复杂网络结构影响复杂网络动态的客观规律,以及有关时空序列可预测性的科学问题。



%%% ++++++++++++++++++++++++++++++++++++++++++++++++++++++++++++++++++++++++++++++++++
